\clearpage
\section{Zusammenfassung und Ausblick}
Es wurde der SE(2)-Filter, sowie nötige Grundlagen der Schätztheorie, über Bayes-Filterung und duale Quaternionen vorgestellt. Gängige Verfahren wie der Partikel- und Kalman-Filter wurden beschrieben. Die Implementientierung des Filters, insbesondere System- und Messmodell, sowie das Progressive Update wurden erläutert. Zuletzt wurden die Ergebnisse der Evaluation des SE(2)-Filters im Vergleich zum Partikelfilter dargestellt. Es konnte kein signifikanter Unterschied zwischen den beiden Schätzmethoden festegestellt werden, dazu müssten mit größere Mengen an Bewegungsschritten des Crawlers getestet werden. Weiter mögliche Erweiterungen sind denkbar. Es können weitere Versuche mit neuen Karten vorgenommen werden. Die quadratische Karte war die einzige, mit der echte Daten gesammelt wurden. Eine komplexere Messfunktion, die die Kegelform des Ultraschallmessfelder berücksichten, würde den Crawler besser modellieren.  Nur aus den vier stark verrauschten Distanzdaten auf die Position des Laufroboters zu schließen ist ein schwieriges Problem. Es wäre interessant zu überprüfen, wie sich die Verfahren verhalten, wenn enger an die Postion gekoppelte Messgrößen beobachtet werden, zum Beispiel durch die IMU des Crawlers. In späteren Versuchen könnte man sich von der Ebene loslösen und die Lokalisation in 3D fortsetzen.